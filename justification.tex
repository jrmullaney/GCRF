\documentclass[11pt]{article}
\usepackage[a4paper,bindingoffset=0cm,%
            left=2cm,right=2cm,top=2cm,bottom=2cm,%
            footskip=0.5cm]{geometry}
\usepackage[pdftex]{graphicx}
\usepackage{multicol}
%\usepackage{psfig}

\usepackage{siunitx,amsmath,amssymb}

\usepackage{sectsty}
\sectionfont{\fontsize{12}{15}\selectfont}
\subsectionfont{\fontsize{11}{15}\selectfont}

%\renewcommand*\sfdefault{phv}
%\renewcommand{\familydefault}{\sfdefault}
\makeatletter
\newenvironment{tablehere}
  {\def\@captype{table}}
  {}

\newenvironment{figurehere}
  {\def\@captype{figure}}
  {}
\makeatother

\begin{document}
%\title{Broadening the impact of astronomical data handling\vspace{-2ex}}
%\maketitle
\setcounter{figure}{0}
\noindent
{\LARGE \bf From Stars to Baht: Broadening the economic impact of \\
astronomical data handling techniques in Thailand.}

\vspace{5mm}

\noindent
{\Large \bf Case for support}

\vspace{2mm}
\noindent

\section{Description of the proposed project}

We request funds to research how the data archiving and analysis techniques developed as part of our STFC/Newton-funded project can be best adapted to address the needs of Thai businesses and organisations. To achieve this research goal, our multi-disciplinary team of Thai and UK scientists will spend 12 months working closely with a group of pre-identified local businesses and organisations based in northen Thailand that have specific data handling and analysis needs.

\subsection{Background: Using astronomy data to train Thai data scientists}
As outlined in our ODA statement, there is strong evidence that improving access to advanced data handling and analysis techniques is one of the most effective ways of increasing productivity in businesses and organisations. As such, equiping Thai students with high-level data handling and analysis skills will play an important role in the continuation of Thailand's economic development.

\vspace{2mm}
\noindent
Fortunately, there already exists in Thailand a knowledgebase in data handling and analysis that is able to teach the next generation of data scientists. The problem, however, is that Thai students typically only have access to small amounts of data. This places severe limits on their ability to develop the skills and experience necessary to analyse the large amounts of digital data generated by modern industries. To address this problem, we successfully applied for Newton funding to train Thai students in databasing and automatically analysing large amounts of astronomical data. The data they are using is being generated by the Gravitational-Wave Optical Transient Observatory (GOTO), of which the PI and some co-Is are members. GOTO surveys the whole night sky every two weeks, delivering data on roughly seven million astronomical sources {\it every night}. This large, constantly-updated dataset is providing Thai data scientists and their students the experience they need to develop the skills and techniques needed to analyse the kind of ``Big Data'' generated by high-tech industries.

\vspace{2mm}
\noindent


\subsection{Broadening the impact of our work: A feasibility study}

\subsection{Description of work to be undertaken}

\subsection{Maximising impact in Thailand}

\subsection{Secondary benefits of the project}

\section{Management plan}

\section{Track record of applicants}


\vspace{10mm}
\noindent{\large\bf References} {\scriptsize 1. Fabian 2012,
  ARA\&A,50,455;
}


\end{document}
