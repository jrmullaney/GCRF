\documentclass[11pt]{article}
\usepackage[a4paper,bindingoffset=0cm,%
            left=2cm,right=2cm,top=2cm,bottom=2cm,%
            footskip=0.5cm]{geometry}
\usepackage[pdftex]{graphicx}
\usepackage{multicol}
%\usepackage{psfig}

\usepackage{siunitx,amsmath,amssymb}

\usepackage{sectsty}
\sectionfont{\fontsize{12}{15}\selectfont}
\subsectionfont{\fontsize{11}{15}\selectfont}

%\renewcommand*\sfdefault{phv}
%\renewcommand{\familydefault}{\sfdefault}
\makeatletter
\newenvironment{tablehere}
  {\def\@captype{table}}
  {}

\newenvironment{figurehere}
  {\def\@captype{figure}}
  {}
\makeatother

\begin{document}
%\title{Broadening the impact of astronomical data handling\vspace{-2ex}}
%\maketitle
\setcounter{figure}{0}
\noindent
{\LARGE \bf From Stars to Baht: Broadening the economic impact of \\
astronomical data handling techniques in Thailand}

\vspace{5mm}

\noindent
{\Large \bf ODA statement}

\vspace{2mm}

\noindent
With a per capita GDP of 5,907~USD in 2016 ({\it Data:} World Bank), Thailand falls firmly within the OECD's DAC list of upper middle income economies. Like most such economies, Thailand has already successfully overcome many of the greatest problems of developing countries, such as basic infrastructure development. Instead, the primary economic challenge that Thailand now faces is to transition from a middle to a high-income economy. Making this transition has proven extremely difficult for many other developing economies, with some remaining stuck in the so-called ``middle-income trap'' for decades (Wold Bank, 2012). Avoiding this trap is crucial if Thailand's recent strong economic growth is to be sustained over the long term.     

\vspace{2mm}
\noindent
The detailed causes and means of avoidance of the middle income trap are complex and subject to debate. However, a common theme among countries that have transitioned from middle to high-income economies during the past 50 years is a sustained increase in productivity (i.e., economic output per hour worked; Aiyar et al, 2013). In recent years it has been recognised that one of the most effective means of increasing productivity is via the scientific analysis of digital data to identify trends in, for example, customer information, patient medical records, distribution networks etc. (Nesta, 2012). There is clear evidence, therefore, that providing Thai businesses and organisations access to the skills needed to handle and analyse large digital datasets will assist Thailand's transition to a high-income economy. This is even more pertinent in the regions outside Bangkok which have traditionally suffered from slower rates of growth (World Bank, 2017), possibly due to poorer access to high-tech skillsets.    

\vspace{2mm}
\noindent
Our team's long-term ambition is to provide Northen Thailand's businesses and organisations with home-grown talent in handling and analysing large quantities of digital data. Northern Thailand is one of the regions that have experienced weaker economic growth compared to the rest of the country (World Bank, 2017). However, as our Letters of Support demonstrate, businesses and organisations in that area are crying out for access to the data handling and analysis skills and technologies that our team can provide. If this demand is not met locally, there is the risk that they and others like them will seek these skills further afield in Bangkok or, worse, from companies not based in Thailand, to the detriment of the local economy.

\vspace{2mm}
\noindent
The first step in achieving our long-term goal has been to give Thai data scientists and their students experience in handling and analysing large, frequently-updated digital datasets of the kind produced by modern industries. This need was met through our Newton-funded project which gave the Thai team members access to large amounts of astronomical data. To date, this has been extremely successful; not only have the students acquired valuable training in ``Big Data'' analysis (as planned) but the team has also developed novel data handling and analysis technologies to overcome some of the challenges presented by this type of data; see our Case for Support. The next step in reaching our long-term goal is for the team to gain experience in working with external partners, which is the motivation for this, our first GCRF funding application. In researching how to adapt and develop our skills and technologies to the needs of our six external partners we will have a real economic impact on parties external to our team. More importantly, it will prepare team members for the next step in our long-term plan: to work with a larger, more diverse range of businesses and organisations (this will form the basis of an application for follow-on GCRF funding). That will give our team the experience necessary to reach our ultimate goal: to set up a self-sustaining ``Centre of Excellence'' for researching technologies to meet the data handling and analysis needs of businesses and organisations in Northern Thailand.

\vspace{10mm}
\noindent{\large\bf References} {\scriptsize 1. Fabian 2012,
  ARA\&A,50,455;
}


\end{document}