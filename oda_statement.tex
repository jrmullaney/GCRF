\documentclass[11pt]{article}
\usepackage[a4paper,bindingoffset=0cm,%
            left=2cm,right=2cm,top=2cm,bottom=2cm,%
            footskip=0.5cm]{geometry}
\usepackage[pdftex]{graphicx}
\usepackage{multicol}
%\usepackage{psfig}

\usepackage{siunitx,amsmath,amssymb}

\usepackage{sectsty}
\sectionfont{\fontsize{12}{15}\selectfont}
\subsectionfont{\fontsize{11}{15}\selectfont}

%\renewcommand*\sfdefault{phv}
%\renewcommand{\familydefault}{\sfdefault}
\makeatletter
\newenvironment{tablehere}
  {\def\@captype{table}}
  {}

\newenvironment{figurehere}
  {\def\@captype{figure}}
  {}
\makeatother

\begin{document}
%\title{Broadening the impact of astronomical data handling\vspace{-2ex}}
%\maketitle
\setcounter{figure}{0}
\noindent
{\LARGE \bf From Stars to Baht: Broadening the economic impact of \\
astronomical data handling techniques in Thailand.}

\vspace{5mm}

\noindent
{\Large \bf ODA statement}

\vspace{2mm}

\noindent
With a per capita GDP of 5,907~USD in 2016 ({\it Data:} World Bank), Thailand falls firmly within the OECD's DAC list of upper middle income economies. Like most such economies, Thailand has already successfully overcome many of the greatest problems of developing countries, such as basic infrastructure development. Instead, the primary economic challenge that Thailand now faces is to transition from a middle to a high-income economy. Making this transition has proven extremely difficult for many other developing economies, with some remaining stuck in the so-called ``middle-income trap'' for decades (Wold Bank, 2012). Avoiding this trap is crucial if Thailand's recent strong rate of economic growth is to be sustained over the long term.     

\vspace{2mm}
The detailed causes and means of avoidance of the middle income trap are complex and subject to debate. However, a common theme among countries that have transitioned from middle to high-income economies during the past 50 years is a sustained increase in productivity (i.e., economic output per hour worked; Aiyar et al, 2013). In recent years it has been recognised that one of the most effective means of increasing productivity is via the scientific analysis of digital data to identify trends in, for example, customer information, patient medical records, distribution networks etc. (Nesta, 2012). There is clear evidence, therefore, that providing Thai businesses and organisations with the skills needed to handle and analyse large digital datasets -- i.e., the primary objective of this project -- will assist Thailand's transition to a high-income economy.

\vspace{10mm}
\noindent{\large\bf References} {\scriptsize 1. Fabian 2012,
  ARA\&A,50,455;
}


\end{document}